\documentclass[pdf]{beamer}

\setbeamercovered{invisible}
\setbeamercovered{again covered={\opaqueness<1->{15}}}

\mode<presentation>{}

\RequirePackage[utf8]{inputenc}
\RequirePackage[T1]{fontenc}

\RequirePackage{color}
\definecolor{codegreen}{rgb}{0,0.6,0}
\definecolor{codegray}{rgb}{0.5,0.5,0.5}
\definecolor{codepurple}{rgb}{0.58,0,0.82}
\definecolor{backcolour}{rgb}{0.95,0.95,0.92}

\RequirePackage{listings}
\lstdefinestyle{python}{
    language = Python,
    commentstyle=\color{codegreen},
    keywordstyle=\color{magenta},
    numberstyle=\tiny\color{codegray},
    stringstyle=\color{codepurple},
    basicstyle=\footnotesize,
    breakatwhitespace=false,
    breaklines=true,
    captionpos=b,
    keepspaces=true,
    showspaces=false,
    showstringspaces=false,
    showtabs=false,
    tabsize=2
}
\lstset{style = Python}

\RequirePackage{siunitx}
\DeclareSIUnit\year{yr}

\RequirePackage{adjustbox}
\RequirePackage{graphics}

\RequirePackage{algorithmicx}

\title{Estimation of computational complexity}
\author{Jørgen Kvalsvik}
\date{September 2018}

\begin{document}
\maketitle

\begin{frame}{Motivation}
    Algorithmic improvements are \emph{the} most impactful changes you can make
    to a program
\end{frame}

\begin{frame}
    \frametitle{Outline}
    \tableofcontents
\end{frame}

\section{Glossary}
\tableofcontents[currentsection]

\begin{frame}
\frametitle{Glossry}
    \begin{block}{Necessary vocab}
        \begin{itemize}
            \item Algorithm
            \item Time and space
            \item Array, vector
            \item Linked list
            \item Dictionary
        \end{itemize}
    \end{block}
\end{frame}

%
% - Best/worst/average case
%
\section{Best/worst/average case}
\tableofcontents[currentsection]

\begin{frame}{Best/worst/average case}
    All three terms describe an algorithm's behaviour under different
    conditions
    \begin{description}
        \item [Best]    Performance under \emph{ideal} input
        \item [Worst]   Performance under \emph{worst} input
        \item [Average] Performance under \emph{any other} input
    \end{description}
\end{frame}

\begin{frame}{Best/worst/average case}
    \begin{itemize}
        \item We use the same vocabulary for \emph{time} and \emph{space}.
        \item \emph{Worst}- and \emph{average} most interesting
    \end{itemize}
\end{frame}


\begin{frame}{Best/worst/average case: example}

    \begin{problem}
        Let $A$ be an unsorted array of length $n$, $f(A, x)$ a function that
        finds $x$ in $A$
    \end{problem}

    \begin{itemize}[<+>]
        \item What adds to execution time (cost)?
        \item How good is the best case?
        \item How bad is the worst case?
        \item How good is the average case?
    \end{itemize}
\end{frame}

\begin{frame}{Best/worst/average case: solution}

    \begin{block}{Solution}
        \begin{description}[<+>]
            \item [Cost]    Access and comparison ($ == $)
            \item [Best]    $x$ is first item, $1$ comparison
            \item [Worst]   $x$ is last item, $n$ comparisons
            \item [Average] $\frac{n}{2}$ comparisons
        \end{description}
    \end{block}

    \uncover<+>{Worst- and average case increase with $n$}
\end{frame}

\section{Histoire d'O}
\tableofcontents[currentsection]

\begin{frame}{Big O notation}
    \begin{definition}[Asymptotic notation]
        Asymptotic, or \emph{big O}, notation is a notation for the worst-case
        \alert{growth} in time/space as a \alert{function of input size}.
    \end{definition}
\end{frame}

\begin{frame}{To infinity and beyond}
    Asymptotic behaviour come into effect when input is \emph{very} large

    \begin{block}{Growth is not affected by}
        \begin{itemize}[<+>]
            \item Constant factors
            \item Lower-order terms
        \end{itemize}
    \end{block}

    \begin{block}{}<+->
        \begin{equation*}
            O(n) = O(n + c) = O(n + log(n))
        \end{equation*}
    \end{block}
\end{frame}

\begin{frame}
    \begin{block}{Note}<+->
        Big O only describes bound on \alert{growth} in the \alert{worst case}.
    \end{block}

    \begin{block}{}<+->
        Some algorithms with better complexity have \emph{worse} behaviour for
        \emph{small} inputs, and inputs are often small
    \end{block}

    \begin{block}{}<+->
        For small inputs, constants often dominate
    \end{block}
\end{frame}

\newcommand{\Ologn}     {$O(log(n))$}
\newcommand{\On}        {$O(n)$}
\newcommand{\Onlogn}    {$O(n \cdot log(n))$}
\newcommand{\Oexp}[1]   {$O(n^{#1})$}

\newcommand{\usec}[1]{\si{\num{#1}\us}}
\newcommand{\msec}[1]{\si{\num{#1}\ms}}
\newcommand{\Sec} [1]{\si{\num{#1}\s}}
\newcommand{\Min} [1]{\si{\num{#1}\min}}
\newcommand{\Hr}  [1]{\si{\num{#1}\hour}}
\newcommand{\Day} [1]{\si{\num{#1}\day}}

\begin{frame}{Compare growth rates}

    \begin{tabular}{c | c c c c c}
        $n$     & $O(1)$    & \Ologn     & \On      & \Onlogn   & \Oexp{2}   \\
        \hline
        $10^{2}$ & \usec{1} & \usec{1}  & \usec{1}  & \usec{1}  & \usec{1}   \\
        $10^{3}$ & \usec{1} & \usec{1.5}& \usec{10} & \usec{15} & \usec{100} \\
        $10^{4}$ & \usec{1} & \usec{2}  & \usec{100}& \usec{200}& \msec{10}  \\
        $10^{5}$ & \usec{1} & \usec{2.5}& \msec{1}  & \msec{2.5}& \Sec{1}    \\
        $10^{6}$ & \usec{1} & \usec{3}  & \msec{10} & \msec{30} & \Min{1.7}  \\
        $10^{7}$ & \usec{1} & \usec{3.5}& \msec{100}& \msec{350}& \Hr{2.8}   \\
        $10^{8}$ & \usec{1} & \usec{4}  & \Sec{1}   & \Sec{4}   & \Day{11.7} \\
    \end{tabular}
\end{frame}

\begin{frame}{Compare growth rates}

    \begin{center}
        \begin{tabular}{c | c c}
            $n$     & \Oexp{2} & $O(2^{n})$         \\
            \hline
            $100$ & \usec{1}   & \usec{1}           \\
            $110$ & \usec{1.2} & \msec{1}           \\
            $120$ & \usec{1.4} & \Sec{1}            \\
            $130$ & \usec{1.7} & \Min{18}           \\
            $140$ & \usec{2}   & \Day{13}           \\
            $150$ & \usec{2.3} & \si{37\year}       \\
            $160$ & \usec{2.6} & \si{37000\year}    \\
        \end{tabular}
    \end{center}
\end{frame}

\begin{frame}{Lesson}
    Programs with \Oexp{2} complexity or worse get impractical \emph{very}
    fast
\end{frame}

\section{Estimating complexity}
\tableofcontents[currentsection]

\begin{frame}
    \begin{block}{Tools}
        \begin{itemize}[<+->]
            \item Ignore constant factors
            \item Eliminate lower cost terms
            \item Consider only what scales with input
            \item Know the cost of \emph{that other stuff}
        \end{itemize}
    \end{block}
\end{frame}

\begin{frame}{That other stuff}
    \only<+->{}
    \begin{tabular}{l | l}
        Array access            & \only<+->{\color[rgb]{0, 0.6, 0}{$O(1)$}}  \\
        Dict access             & \only<+->{\color[rgb]{0, 0.6, 0}{$O(1)$}}  \\
        Dict insertion          & \only<+->{\color[rgb]{0, 0.6, 0}{$O(1)$}}  \\
        Array insertion back    & \only<+->{\color[rgb]{0, 0.6, 0}{$O(1)$}}  \\
        Search sorted list      & \only<+->{\color[rgb]{0, 0.6, 0}{\Ologn}}  \\
        Search unsorted list    & \only<+->{\color[rgb]{0, 0.6, 0}{\On}}     \\
        Array insertion front   & \only<+->{\color[rgb]{0, 0.6, 0}{\On}}     \\
        Sorting                 & \only<+->{\color[rgb]{0, 0.6, 0}{\Onlogn}} \\
    \end{tabular}
\end{frame}

\begin{frame}{Estimation}
    \begin{block}{Count loops}<+->
        \begin{itemize}[<+->]
            \item Every loop is another order of $n$
            \item First loop is \On, second is \Oexp{2}
            \item Nesting is \emph{multiplicative}
        \end{itemize}
    \end{block}

    \uncover<+>{
        Every new phase is \emph{additive}
    }
\end{frame}

\begin{frame}{Ding-ding-ding}
    \uncover<+->{Highest term wins}
    \uncover<+>{
        \begin{equation*}
            O(n^{2} + n + c) = O(n^{2})
        \end{equation*}
    }
\end{frame}

\begin{frame}[fragile]{Loop counter}
    \begin{lstlisting}[style = Python]
        def cartesian(xs, ys):
            cart = []
            for x in xs:
                for y in ys:
                    cart.append((x, y))

            return cart
    \end{lstlisting}

    What's the complexity?
\end{frame}

\begin{frame}{Breakdown}
    \begin{tabular}{l | l}
        \lstinline!cart = []!   & \uncover<2,6->{$O(1)$}    \\
        \lstinline!for x in xs! & \uncover<3,6->{$O(n)$}    \\
        \lstinline!for y in ys! & \uncover<4,6->{$O(m)$}    \\
        \lstinline!cart.append! & \uncover<5,6->{$O(1)$}    \\
    \end{tabular}

    \begin{block}{Total}
        \begin{equation*}
            \onslide<1->{O(}
                \onslide<2->{1}
                \onslide<3->{+ n}
                \onslide<4->{\cdot n}
                \onslide<5->{\cdot 1}
                \onslide<1->{)}
            \onslide<6->{= O(nm)}
        \end{equation*}
    \end{block}

    \uncover<7->{In practice \Oexp{2}}
\end{frame}

\begin{frame}[fragile]{Sort in front of cart}
    \begin{lstlisting}[style = Python]
        xs = read_input()
        cart1 = cartesian(xs, xs)
        xs.sort()
        cart2 = cartesian(xs, xs)
    \end{lstlisting}

    What's the complexity?
\end{frame}

\begin{frame}{Breakdown}
    \begin{tabular}{l | l}
        \lstinline!read_input!  & \uncover<2,6->{$O(n)$}        \\
        \lstinline!cartesian!   & \uncover<3,6->{$O(n^{2})$}    \\
        \lstinline!xs.sort!     & \uncover<4,6->{$O(nlog(n))$}  \\
        \lstinline!cartesian!   & \uncover<5,6->{$O(n^{2})$}    \\
    \end{tabular}

    \begin{block}{Total}
        \begin{equation*}
            \onslide<1-> {O(}
                \onslide<2->{n}
                \only<3-5>{+ n^{2}}
                \onslide<4->{ + n \cdot log(n)}
                \only<5>{    + n^{2}}
                \only<6->{    + 2 n^{2}}
                \onslide<1->)
            \onslide<7-> = O(n^{2})
        \end{equation*}
    \end{block}

    \uncover<7-> Highest term wins
\end{frame}

\section{Choosing the worse algorithm}
\tableofcontents[currentsection]

\begin{frame}{Be quick or be merged}
    \begin{block}{}<+>
        Some algorithms are worse on paper, but perform better in practice
    \end{block}

    \begin{block}{Sorting}<+->
        \begin{adjustbox}{max width=\textwidth}

            \begin{tabular}{c | c c c c}
                Algorithm & Time (best) & Time (worst) & Time (avg) & Space  \\
                \hline
                Quicksort & \Onlogn     & \Oexp{2}     & \Onlogn    & \On    \\
                Mergesort & \Onlogn     & \Onlogn      & \Onlogn    & \On    \\
                Heapsort  & \Onlogn     & \Onlogn      & \Onlogn    & $O(1)$ \\
                Insertion & \On         & \Oexp{2}     & \Oexp{2}   & $O(1)$ \\
            \end{tabular}

        \end{adjustbox}
    \end{block}

    \begin{block}{}<+->
        Why don't we always heapsort?
    \end{block}
\end{frame}

\let\oldquote\quote
\let\endoldquote\endquote
\renewenvironment{quote}[2][]{

    \if\relax\detokenize{#1}\relax
       \def\quoteauthor{#2}%
    \else
        \def\quoteauthor{#2~---~#1}%
    \fi

    \oldquote
}{
    \par\nobreak\smallskip\hfill\quoteauthor%
    \endoldquote\addvspace{\bigskipamount}
}

\begin{frame}{Shoulders of giants}
    \begin{block}{Quote of the day}<+->
        \begin{quote}{Donald Knuth}
            Premature optimisation is the root of all evil
        \end{quote}
    \end{block}

    \begin{block}{}<+->
        Write clear code - don't lose a $O(n^{3}) \rightarrow O(n \cdot
        log(n))$ optimisation chasing faster array access
    \end{block}
\end{frame}

\begin{frame}{Shoulders of giants}
    \begin{block}{Don't forget constants}<+>
        \begin{align*}
            O(0.01 n^{3} + 10 n + 5) &= O(n^{3}) \\
            O(2000 n^{2} + 10)       &= O(n^{2})
        \end{align*}
    \end{block}
\end{frame}

\begin{frame}{A note on growth}
    \includegraphics[width=\textwidth]{growth1.png}
\end{frame}

\begin{frame}{A note on growth}
    \includegraphics[width=\textwidth]{growth2.png}
\end{frame}

\begin{frame}[fragile]{Repeated-filter vs composed-filter}
    \begin{columns}
        \column{0.5\textwidth}
        \begin{block}{Sequential}
            \begin{algorithmic}
                \State $xs \gets input$
                \State $x \gets f(xs)$
                \State $y \gets h(x)$
                \State $z \gets g(y)$
            \end{algorithmic}
        \end{block}

        \column{0.5\textwidth}
        \begin{block}{Composed}
            \begin{algorithmic}
                \State $xs \gets input$
                \State $f' \gets g \circ h \circ f$
                \State $z \gets f'(xs)$
            \end{algorithmic}
        \end{block}
    \end{columns}

    \begin{block}{}<+>
        Composed makes fewer passes over data
    \end{block}
\end{frame}

\section{Summary}
\tableofcontents[currentsection]

\begin{frame}
    \begin{itemize}[<+->]
        \item Know the cost of basic operations
        \item Count nested loops
        \item Prefer clear code
        \item Two linear programs may behave \emph{very} differently
        \item Complexity is \emph{not} sufficient to estimate performance
    \end{itemize}
\end{frame}

\end{document}
