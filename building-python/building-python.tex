\documentclass[pdf]{beamer}

\RequirePackage[utf8]{inputenc}
\RequirePackage[T1]{fontenc}
\RequirePackage{lmodern}

% for speaker notes etc
\RequirePackage{pgfpages}
\setbeameroption{show notes on second screen}
%\setbeameroption{show only notes}
\setbeamercolor{note page}{bg=white}
\setbeamercolor{note title}{bg=white!90!black, fg=black}
\setbeamercolor{note date}{parent=note title}

\beamertemplatenavigationsymbolsempty
\AtBeginSection[]{
    \begin{frame}
        \vfill
        \centering
        \begin{beamercolorbox}[sep=8pt,center,shadow=true,rounded=true]{title}
            \usebeamerfont{title}\insertsectionhead\par%
        \end{beamercolorbox}
        \vfill
    \end{frame}
}

\RequirePackage{listings}
\lstset{escapeinside={<@}{@>}}
\lstset{basicstyle=\ttfamily}
\lstset{moredelim=**[is][\color{red}]{@}{@}}

\RequirePackage{algpseudocode}

\mode<presentation>{}

\title{Building C++-Python libraries}
\subtitle{}
\author{Jørgen Kvalsvik <jokva@equinor.com>}
\titlegraphic{\includegraphics[width=0.33\textwidth]{equinor-red.eps}}

\begin{document}
\maketitle

\begin{frame}{Outline}
    \tableofcontents

    \note{
        \begin{itemize}
            \item something python api for users
        \end{itemize}
    }
\end{frame}

\begin{frame}{Types of Python}
    \begin{itemize}
        \item Pure python
        \item Impure Python (non-python code behind CPython API)
        \item Thin library wrappers
        \item Fat library wrappers
    \end{itemize}

    \note{
        \begin{itemize}
            \item Pure python is what we want out downstream users to do
            \item Impure python is almost indistinguishable from pure python.
                  Numpy and chunks of the python standard library are good
                  examples.
            \item Thin wrappers rely on some system provided library and
                  provide an entry point from Python. Ctypes is often use for
                  this - fire, wait, and parse result
            \item Rely on larger, often system-wide external libraries, but do
                  more than just provide an entry point - sophisticated I/O,
                  manipulation of state, and rich access to library
                  functionality. Difference between this and impure python is
                  that numpy's backend is only available in Python, contrary to
                  say python-BLAS

        \end{itemize}
        We'll focus on impure and fat library.
    }
\end{frame}

\begin{frame}{Some goals}
    python3 -m pip install pkg
\end{frame}

\begin{frame}{pip install}
    What does pip install do?
    \note{
    }
\end{frame}


\begin{frame}{\emph{Building} interpreted packages 1/2}
    Python packages are just source code, parsed and interpreted on-the-fly

    \note{
        \begin{itemize}
            \item We still talk about \emph{building} packages even though it's
                  a glorified directory copy
        \end{itemize}
    }
\end{frame}

\begin{frame}[fragile]{Directory layout 1/2}
    \begin{verbatim}
README.rst
LICENSE
setup.py
requirements.txt
package/__init__.py
package/core.py
package/helpers.py
docs/conf.py
docs/index.rst
tests/test_basic.py
tests/test_advanced.py
    \end{verbatim}

{\tiny Source: }

{\tiny https://www.kennethreitz.org/essays/repository-structure-and-python}

{\tiny https://docs.python-guide.org/writing/structure}
    \note{
        \begin{itemize}
            \item Building this package is essentially just copying the
                  package/ directory to the correct python/site-packages
            \item Sometimes you want to do some pre-processing to source files,
                  pre-set variables depending on system etc
        \end{itemize}
    }
\end{frame}

\begin{frame}[fragile]{Directory layout 2/2}
    \begin{verbatim}
README.rst
LICENSE
setup.py
requirements.txt
src/package/__init__.py
src/package/core.py
src/package/helpers.py
docs/conf.py
docs/index.rst
tests/test_basic.py
tests/test_advanced.py
    \end{verbatim}

    \note{
        \begin{itemize}
            \item Notice the src/ folder
            \item Can have multiple packages in the same tree
            \item Has a few benefits, no implicit import of source code
        \end{itemize}
    }
\end{frame}

\begin{frame}{\emph{Building} interpreted packages 2/2}
    Building is the transformation from a source-tree to something predictably
    laid out in site-packages

    \note{
        \begin{itemize}
            \item For this, \emph{if} you conform to a common project layout,
                  python tooling works reasonably well.
            \item How did we get here?
        \end{itemize}
    }
\end{frame}

\begin{frame}{Some history}
    \begin{description}
        \item [Late 90s] Python wanted something like CPAN
        \item [2000] Distutils released for Python 1.6 standard library, where
                     modules go to die
        \item [2003] PyPI is online, distutils can create package metadata
        \item [2008] Setuptools replaces distutils, pip builds on it
    \end{description}

    \note{
        \begin{itemize}
            \item Distutils came with infrastructure for building C code
            \item Feels much intended for building Python extensions, by having
                  the same compiler that built Python.
            \item This is still common on unix-like systems, terrible on
                  Windows where dealing with compilers is a pain.
            \item Uses homegrown compiler abstraction and option dispatch
                  interface. Little robust auto discovery, half-way to cmake
            \item Honestly, setup.py-files that do more than setup() tend to be
                  low-effort, buggy, "works on my machine" solutions. Robust
                  files get complicated \emph{fast}.
            \item Pollutes source tree
        \end{itemize}
    }
\end{frame}

\begin{frame}[fragile]{Building native code}
    The struggle of binary libraries

    \begin{itemize}
        \item Multiple compilers with different flags
        \item Compiler versions
        \item Multiple available compilers
        \item Platform-specific options
        \item Configuration-specific options, e.g. \verb|USE_BLAS|
        \item Feature detection
        \item Build and runtime dependencies
        \item Binary compatibility
    \end{itemize}

    \note{
        Read list

        \begin{itemize}
            \item The experience compiling and developing native extensions is
                  terrible
            \item Setuptools has virtually nothing to help you here, boils down
                  to manually implementing half of cmake
            \item The development story is quite bad too
            \item People end up hard-coding links top /opt/local/lib64, I can't
                  blame them
        \end{itemize}
    }
\end{frame}

\begin{frame}{Developing native code}
    \begin{itemize}
        \item Changes in headers aren't detected, and the module isn't
              recompiled. Not great when you have templates
        \item Not designed for interactive development, desgined for
              distribution
        \item Setuptools assumes it controls the world, so it pollutes build
              trees. Not cool when python support is a sub project
    \end{itemize}

    \note{
        \begin{itemize}
            \item Not designing for live development of C based modules is
                  fine, they weren't aiming to build a C build system. But most
                  time is spent is development (by me), and having parallel
                  development and build systems suck
        \end{itemize}
    }
\end{frame}

\begin{frame}
    Is all hope lost?

    \note{
        \begin{itemize}
            \item The talk name did hint of a solution
        \end{itemize}
    }
\end{frame}

\end{document}
