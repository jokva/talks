\documentclass[pdf]{beamer}

\RequirePackage[utf8]{inputenc}
\RequirePackage[T1]{fontenc}
\RequirePackage{lmodern}

% for speaker notes etc
\RequirePackage{pgfpages}
\setbeameroption{show notes on second screen}
%\setbeameroption{show only notes}
\setbeamercolor{note page}{bg=white}
\setbeamercolor{note title}{bg=white!90!black, fg=black}
\setbeamercolor{note date}{parent=note title}

\beamertemplatenavigationsymbolsempty
\AtBeginSection[]{
    \begin{frame}
        \vfill
        \centering
        \begin{beamercolorbox}[sep=8pt,center,shadow=true,rounded=true]{title}
            \usebeamerfont{title}\insertsectionhead\par%
        \end{beamercolorbox}
        \vfill
    \end{frame}
}

\RequirePackage{listings}
\lstset{escapeinside={<@}{@>}}
\lstset{basicstyle=\ttfamily}
\lstset{moredelim=**[is][\color{red}]{@}{@}}

\RequirePackage{dirtree}
\RequirePackage{csquotes}

\RequirePackage{listings}
% use monospaced fonts in listings
\lstset{basicstyle=\ttfamily\scriptsize}

\mode<presentation>{}

\title{Building C++-Python libraries}
\subtitle{}
\author{Jørgen Kvalsvik <jokva@equinor.com>}
\titlegraphic{\includegraphics[width=0.33\textwidth]{equinor-red.eps}}

\begin{document}
\maketitle

\begin{frame}{Outline}
    \tableofcontents

    \note{
        \begin{itemize}
            \item something python api for users
        \end{itemize}
    }
\end{frame}

\begin{frame}{Types of Python}
    \begin{itemize}
        \item Pure python
        \item Impure Python (non-python code behind CPython API)
        \item Thin library wrappers
        \item Fat library wrappers
    \end{itemize}

    \note{
        \begin{itemize}
            \item Pure python is what we want out downstream users to do
            \item Impure python is almost indistinguishable from pure python:
                  Numpy and chunks of the python standard library are good
                  examples.
            \item Thin wrappers rely on some system provided library and
                  provide an entry point from Python. Ctypes is often use for
                  this - fire, wait, and parse the output
            \item Rely on larger, often system-wide external libraries, but do
                  more than just provide an entry point - sophisticated I/O,
                  manipulation of state, and rich access to library
                  functionality. Difference between this and impure python is
                  that numpy's backend is only available in python, contrary to
                  say python-BLAS
        \end{itemize}
        We'll focus on impure and fat library.
    }
\end{frame}

\begin{frame}{A goal}
    python3 -m pip install my-project
\end{frame}

\begin{frame}{pip install}
    What does pip install do?

    \begin{enumerate}
        \item Looks up the package name in an \emph{index}
        \item Determines if the package is installable on this system
        \item Fetches the package
        \item Extracts the contents to the right site-package directory
    \end{enumerate}
\end{frame}

\begin{frame}
    \begin{center}
    \includegraphics[keepaspectratio, height = 0.9\textheight]{snake.jpg}
    \end{center}
    \note {
        This is a Python package
    }
\end{frame}

\begin{frame}{\emph{Building} interpreted packages 1/2}
    Python packages are just source code, parsed and interpreted on-the-fly

    \note{
        \begin{itemize}
            \item We still talk about \emph{building} packages even though it's
                  a glorified directory copy
            \item To shed some light on it, let's look at the internals of your
                  run-of-the-mill python project
        \end{itemize}
    }
\end{frame}

\begin{frame}{Directory layout 1/3}
    \dirtree{%
        .1 project/.
        .2 README.rst.
        .2 LICENSE.
        .2 setup.py.
        .2 requirements.txt.
        .2 package/.
        .3 {\_\_}init{\_\_}.py.
        .3 core.py.
        .3 helpers.py.
        .2 docs/.
        .3 conf.py.
        .3 index.rst.
        .2 tests/.
        .3 test{\_}basic.py.
        .3 test{\_}advanced.py.
    }

    {\tiny https://www.kennethreitz.org/essays/repository-structure-and-python}

    {\tiny https://docs.python-guide.org/writing/structure}
    \note{
        \begin{itemize}
            \item This is a source tree listing, a git checkout or similar
            \item Building this package is essentially just copying the
                  package/ directory to the correct python/site-packages
            \item Sometimes you want to do some pre-processing to source files,
                  pre-set variables depending on system, bundle data files etc
        \end{itemize}
    }
\end{frame}

\begin{frame}{Directory layout 2/3}
    \dirtree{%
        .1 project/.
        .2 README.rst.
        .2 LICENSE.
        .2 setup.py.
        .2 requirements.txt.
        .2 src/.
        .3 package/.
        .4 {\_\_}init{\_\_}.py.
        .4 core.py.
        .4 helpers.py.
        .2 docs/.
        .3 conf.py.
        .3 index.rst.
        .2 tests/.
        .3 test{\_}basic.py.
        .3 test{\_}advanced.py.
    }

    \note{
        \begin{itemize}
            \item Notice the src/ folder
            \item Can have multiple packages in the same tree
            \item Has a few benefits, such as no implicit import of source code
        \end{itemize}
    }
\end{frame}

\begin{frame}{Directory layout 3/3}
    \dirtree{%
        .1 site-packages/.
        .2 my-package/.
        .3 {\_\_}init{\_\_}.py.
        .3 core.py.
        .3 helpers.py.
        .2 numpy/.
        .2 pandas/.
        .2 minisat/.
    }

    \note{
        \begin{itemize}
            \item Both the previous examples, when built and install, end up
                  like this
        \end{itemize}
    }
\end{frame}

\begin{frame}{\emph{Building} interpreted packages 2/2}
    Building is the transformation from a source-tree to something predictably
    laid out in site-packages

    \note{
        \begin{itemize}
            \item For this, \emph{if} you conform to a common project layout,
                  python tooling works reasonably well.
            \item How did we get here?
        \end{itemize}
    }
\end{frame}

\begin{frame}{Some history}
    \includegraphics[width=\textwidth]{pantheon.jpeg}
    \note{
        \begin{itemize}
            \item Anyone know what this is?
            \item Important to understand how we got here
        \end{itemize}
    }
\end{frame}


\begin{frame}{Some history}
    \begin{description}
        \item [Late 90s] Python wanted something like CPAN
        \item [2000] Distutils released for Python 1.6 standard library
        \item [2003] PyPI is online, distutils can create package metadata
        \item [2008] Setuptools extends and supersedes distutils, pip builds on it
    \end{description}

    \note{
        \emph{Read the list}: standard library is where modules go to die
        \begin{itemize}
            \item Notice \emph{everything builds on}
            \item Setup written for distutils generally work with setuptools,
                  strong backwards compatibility. This also means a lot of
                  legacy
            \item Distutils came with infrastructure for building C code
            \item Feels much intended for building Python extensions, using the
                  same compiler that built Python
            \item This is terrible on Windows where dealing with compilers is a
                  pain
            \item Uses homegrown compiler abstraction and option dispatch
                  interface
            \item No custom options really, stuck on the configuration
                  setuptools already has
            \item Building pollutes source tree
            \item Let's start building a package
        \end{itemize}
    }
\end{frame}

\begin{frame}[fragile]{setuptools}
    \lstinputlisting{my-package/setuptools-pure.py}

    \note{
        \begin{itemize}
            \item This is for a pure python package
            \item It's quite minimal, setuptools support a wide range of keys
                  and metadata. It can figure some things out, since our
                  project is "standard layout".
            \item Let's add some native code, a C extension
        \end{itemize}
    }
\end{frame}

\begin{frame}[fragile]{setuptools}
    \lstinputlisting{my-package/my_package/math/math.c}
\end{frame}

\begin{frame}[fragile]{setuptools}
    \lstinputlisting{my-package/setuptools-ext.py}
\end{frame}

\begin{frame}[fragile]{setuptools + extension}
    \begin{lstlisting}
>>> import my_package
>>> my_package.math.add(1, 2)
3
    \end{lstlisting}
    \note {
        \begin{itemize}
            \item This extension is, of course, trivial
            \item Let's spice it up and use C++11
        \end{itemize}
    }
\end{frame}

\begin{frame}[fragile]{setuptools}
    \lstinputlisting{my-package/my_package/math/math11.cpp}
    \note{
        \begin{itemize}
            \item This now fails on gcc < 6.0, because C++98 was still the
                  default
            \item Of course it's still valid C++, so it just needs the option
        \end{itemize}
    }
\end{frame}

\begin{frame}[fragile]{setuptools}
    \lstinputlisting{my-package/setuptools-ext11.py}

    \note{
        \begin{itemize}
            \item Ok good, so this builds... on linux
            \item MSVC doesn't understand this flag
            \item So now our script needs to detect compiler and only set the
                  flag on gcc
            \item ... until we also want to build with clang
            \item If you've ever built C++ for multiple platforms or even
                  different computers you see how this can get problematic fast
            \item It's quite hard to ad-hoc configure on new systems
        \end{itemize}
    }
\end{frame}

\begin{frame}[fragile]{setuptools}
    \begin{verbatim}
# setuptools on microsoft compilers doesn't support the --library-dir or
# --build-dir flag and crashes, so only pass it on non-microsoft platforms
    \end{verbatim}
    \note{
        \begin{itemize}
            \item This is a snippet from our project segyio
            \item setuptools actually \emph{crashed} on being given
                  --library-dirs
            \item Ok, so maybe not pass it when you're sitting on windows, but
                  what about when this is in scripts
            \item Honestly, setup.py-files that do more than setup() and are
                  pure python tend to be low-effort, buggy, \emph{works on my
                  machine} solutions
            \item At this point we give up
        \end{itemize}
    }
\end{frame}

\begin{frame}[fragile]{Building native code}
    The struggle of binary libraries

    \begin{itemize}
        \item Multiple compilers with different flags
        \item Compiler versions
        \item Multiple available compilers
        \item Platform-specific options
        \item Configuration-specific options, e.g. \verb|USE_BLAS|
        \item Feature detection
        \item Build and runtime dependencies
        \item Binary compatibility
    \end{itemize}

    \note{
        \emph{Read list}
        \begin{itemize}
            \item The experience compiling and developing native extensions is
                  terrible
            \item Setuptools has virtually nothing to help you here, boils down
                  to manually implementing half of cmake
            \item The development story is quite bad too
            \item People end up hard-coding links to /opt/local/lib64, I can't
                  blame them
            \item Since setup.py implements build, install, test commands,
                  custom option commands are difficult or clumsy
        \end{itemize}
    }
\end{frame}

\begin{frame}{Developing native code}
    \begin{itemize}
        \item Changes in headers aren't detected, and the module isn't
              recompiled - not great with templates
        \item Designed for distribution, not development
        \item Setuptools assumes it controls the world, so it pollutes source
              trees. Not cool when python support is a sub project
    \end{itemize}

    \note{
        \begin{itemize}
            \item Overall, setuptools does a poor job of tracking file ->
                  compiled object dependencies
            \item Not designing for live development of C based modules is
                  fine, they weren't aiming to build a C build system. But most
                  time is spent is development (by me), and having parallel
                  development and build systems suck
        \end{itemize}
    }
\end{frame}

\begin{frame}{Is paradise lost?}
    \begin{center}
    \includegraphics[keepaspectratio, height = 0.8\textheight]{fall-of-lucifer.jpg}
    \end{center}
    \note{
        \begin{itemize}
            \item The talk name did hint of a solution
            \item A lot of installed stuff is happy to hard-code path to python
                exe + libs
        \end{itemize}
    }
\end{frame}

\begin{frame}{Enter scikit-build}
    \begin{displayquote}
        The scikit-build package is fundamentally just glue between the
        setuptools Python module and CMake
    \end{displayquote}
    - scikit-build readme

    \note{
        \begin{itemize}
            \item Briefly, it automates building C++ with cmake, and automates
                  dealing with python flags, layout, options.
            \item It adresses most of the shortcomings with setuptools for
                  extensions
            \item For distribution: feature autodiscovery, compiler config,
                  dependency configuration etc
            \item For development: easy debug/release builds, file-object
                  dependency tracking, incremental builds
            \item Setuptools does what setuptools can (build python), cmake
                  does what cmake does
        \end{itemize}
    }
\end{frame}

\begin{frame}[fragile]{CMake}
    cmake gives us:
    \begin{itemize}
        \item \verb|find_package|
        \item \verb|target_link_libraries|
        \item \verb|set(CMAKE_CXX_STANDARD)|
        \item \verb|if (MSVC)|
    \end{itemize}

    \note{
        \begin{itemize}
            \item Hate it or love it, cmake is pretty good at what it does
            \item It's reasonably easy to set options, deal with compiler
                  differences in cmake
            \item When you have third-party dependencies that provide
                  cmake-config or similar, you can easily get them from inside
                  the python extension with find-package
            \item cmake is pretty good at dealing with platform differences
        \end{itemize}
    }
\end{frame}

\begin{frame}[fragile]{scikit-build}
    \lstinputlisting{my-package/CMakeLists.txt}

    \note {
        \begin{itemize}
            \item This is really it. It works quite well, and the maintainers
                  are helpful
            \item Ok, but let's pretend we want to also link against zlib
        \end{itemize}
    }
\end{frame}

\begin{frame}[fragile]{scikit-build}
    \lstinputlisting{my-package/CMakeLists-zlib.txt}
    \note{
        \begin{itemize}
            \item All we had to do was find-package and target-link-libraries
            \item We outsource the dealing with platforms, linking and
                  detection to cmake
            \item The python build script is unaware - this is all a detail
                  of the extension anyway
            \item Let's assume that on the dev machine, zlib is installed in a
                  non-standard location
        \end{itemize}
    }
\end{frame}

\begin{frame}[fragile]{scikit-build}
    \verb|python3 setup.py build -- -DZLIB_DIR=~/.local/share/zlib|

    \note {
        \begin{itemize}
            \item scikit-build allows easy access to cmake configuration
            \item unlike setuptools before, because scikit-build actually
                  extended argument parsing
            \item Now we can easily configure the library without changing
                  build script code
        \end{itemize}
    }
\end{frame}

\begin{frame}[fragile]{Intermezzo: cmake integration}

    \verb|mkdir build && cd build && cmake ..|
    \verb|make -j8 && ctest --output-on-failure|

    \note{
        \begin{itemize}
            \item This command is really what I mostly do during development
            \item It builds the core libraries, some applications, the python
                  extension, the python library, and docs if enabled
            \item And runs the tests, across all languages
            \item It uses relies on setup.py to build and test python
        \end{itemize}
    }
\end{frame}


\begin{frame}[fragile]{Intermezzo: cmake integration}
    \begin{lstlisting}
add_custom_target(
    dlisio-python ALL
    SOURCES ${setup.py}
    DEPENDS ${setup.py}
    VERBATIM
    WORKING_DIRECTORY ${CMAKE_CURRENT_SOURCE_DIR}
    COMMAND ${python} ${setup.py}
        build_ext --inplace
        build # setup.py build args
            --cmake-executable ${CMAKE_COMMAND}
            --generator ${CMAKE_GENERATOR}
            ${DLISIO_PYTHON_BUILD_TYPE}
        -- # scikit-build cmake args
            -Ddlisio_DIR=${DLISIO_LIB_BINARY_DIR}
            -DCMAKE_INSTALL_RPATH_USE_LINK_PATH=ON
            -DCMAKE_INSTALL_RPATH=$<TARGET_FILE_DIR:dlisio>
            -DCMAKE_INSTALL_NAME_DIR=$<TARGET_FILE_DIR:dlisio>
)
add_dependencies(dlisio-python dlisio)
    \end{lstlisting}

\note{
    \begin{itemize}
        \item This is taken from dlisio
        \item If you're not familiar with cmake or custom targets, this might
              be confusing, but it makes so the python lib behaves like any
              other C++ lib from cmake's perspective
        \item Build in-place so that tests can be run directly on the source
              tree
        \item setuptools pollutes this directory anyway, so we're not
              really that much worse off
        \item Go the path of least resistance, even if it is ridiculous
        \item If you fight against the tool you simply won't win
        \item Scikit-build builds for different python versions in different
              dirs, so they coexist ok
        \item setup.py tracks the python dependencies - the drawback is even
              though there's nothing to do we still must run the seutp.py file.
              This is somewhat slow, but an acceptable tradeoff
    \end{itemize}
}
\end{frame}

\begin{frame}[fragile]{Building wheels}
    \begin{block}{On Debian 10 (Buster)}
        \begin{verbatim}
$ python3 setup.py bdist_wheel
$ ls dist/
my_project-0.0.1-cp37-cp37m-linux_x86_64.whl
        \end{verbatim}
    \end{block}

    \note{
        \begin{itemize}
            \item Notice the tag says linux. Not well suited for distribution
                  because users might have older libc
            \item For good binaries we need the manylinux tag
        \end{itemize}
    }
\end{frame}

\begin{frame}[fragile]
    \begin{block}{}
        \begin{itemize}
            \item An ABI tag helps pip select the right package
            \item pip resolves these names from the host system
        \end{itemize}
    \end{block}

    \begin{block}{From PyPI}
        \begin{verbatim}
segyio-1.8.6-cp37-cp37m-manylinux1_x86_64.whl (87.5 kB)
segyio-1.8.6-cp37-cp37m-win32.whl (83.4 kB)
segyio-1.8.6-cp37-cp37m-win_amd64.whl (89.7 kB)
        \end{verbatim}
    \end{block}

    \note{
        \begin{itemize}
            \item \emph{Read list on screen}
            \item OS X was omitted because it's quite long, but follows the
                  same pattern
            \item The manylinux image is specified by the PyPA, basically the
                  oldest libc + supporting libs they could use (RHEL5)
            \item Windows packages built on Windows, linux on RHEL5 etc.
        \end{itemize}
    }
\end{frame}

\begin{frame}
    Problem: I have no machine running Windows
\end{frame}

\begin{frame}
    Problem: I have no machine running Red Hat 5
\end{frame}

\begin{frame}{Build automation}
    \begin{block}{Lesson}
        Things not done automatically will not be done
    \end{block}
\end{frame}

\begin{frame}{Build automation}
    machine definitions + scripts + triggers
    \note{
        \begin{itemize}
            \item A machine definition would be perfect for obsolete systems
                  like RHEL5 and Windows
            \item Multiple machine definitions means we can test on multiple
                  systems
            \item With triggers, they can now run on every PR and every HEAD
                  pushed
            \item A new release can also be a trigger
            \item Makes it easier for contributers too, they don't have to set
                  up a complicated rig to run the new changes on multiple
                  systems
            \item So what does it look like?
        \end{itemize}
    }
\end{frame}

\begin{frame}{Build-and-test PR 1/2}
    \includegraphics[width = \textwidth]{github-green-checkmark.png}
\end{frame}

\begin{frame}{Build-and-test PR 2/2}
    \includegraphics[width = \textwidth]{github-checks-passed.png}
\end{frame}

\begin{frame}{Build automation}
    \begin{center}
    \includegraphics[keepaspectratio, height = 0.8\textheight]{olympic-titanic.jpg}
    \end{center}
\end{frame}

\begin{frame}{Deploy}
    \includegraphics[width = \textwidth]{travis-dlisio-jobs.png}
    \note {
        \begin{itemize}
            \item I mentioned a trigger was a tag
            \item On tags, our deploy logic is run, which as its final step
                  uploads the freshly-built wheels
            \item This means the released archives are never built on developer
                  machines
            \item And we have a one-click deploy feature in place
            \item It's great for velocity, and it means \emph{any} team member
                  can deploy new releases
        \end{itemize}
    }
\end{frame}

\begin{frame}[fragile]{Deploy}
    \begin{lstlisting}
deploy:
    - provider: pypi
      skip_cleanup: true
      skip_upload_docs: true
      skip_existing: true
      user: statoil-travis
      distributions: --skip-cmake build sdist
      password:
        secure: encrypted-key
      on:
        tags: true
    \end{lstlisting}
\end{frame}

\begin{frame}
    \note {
        \begin{itemize}
            \item There's no zlib on the user machine
            \item We'd still like the pip-package to \emph{just work}
            \item In particular an issue for fat wrappers around not-always
                  distributed libraries
        \end{itemize}
    }
\end{frame}

\begin{frame}
    Solution: distribute zlib with the wheel

    \note {
        \begin{itemize}
            \item There's no zlib on the user machine
        \end{itemize}
    }
\end{frame}

\begin{frame}[fragile]
    \begin{block}{On Debian 10 (Buster)}
        \begin{lstlisting}
$ auditwheel repair dist/my_project-0.0.1.whl
Fixed-up wheel written to wheelhouse/my_project-0.0.1.whl
$ unzip -l wheelhouse/my_project-0.0.1.whl
Archive:  wheelhouse/my_project-0.0.1.whl
  Length   Name
---------  ----
       19  my_package/__init__.py
        0  my_package/core.py
    34432  my_package/math.cpython-35m-x86_64-linux-gnu.so
  2646632  my_package/.libs/libstdc++-4376550d.so.6.0.22
  3820672  my_package/.libs/libc-2-d37be6b0.24.so
  1072968  my_package/.libs/libm-2-bfba5a97.24.so
   101072  my_package/.libs/libgcc_s-41124e5d.so.1
   112496  my_package/.libs/libz-7fd423a0.so.1.2.8
     1072  my_project-0.0.1.dist-info/RECORD
       11  my_project-0.0.1.dist-info/top_level.txt
      173  my_project-0.0.1.dist-info/METADATA
       98  my_project-0.0.1.dist-info/WHEEL
---------  --------
  7789645  12 files
        \end{lstlisting}
    \end{block}
    \note {
        \begin{itemize}
            \item There's a similar tool for macos, delocate
            \item And there you have it
            \item All that's left now is
        \end{itemize}
    }
\end{frame}

\begin{frame}[fragile]
    \verb|python3 -m pip install my-project|
\end{frame}

\begin{frame}
    \begin{block}{Shoutouts}
        \begin{description}
            \item [scikit-build] Python and C++ build system
            \item [multibuild] Matthew Brett's wheel building automation
        \end{description}
    \end{block}
\end{frame}

\end{document}
