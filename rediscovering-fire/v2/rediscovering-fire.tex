\documentclass[pdf]{beamer}

\RequirePackage[utf8]{inputenc}
\RequirePackage[T1]{fontenc}
\RequirePackage{lmodern}

% for speaker notes etc
\RequirePackage{pgfpages}
%\setbeameroption{show notes on second screen}
%\setbeameroption{show only notes}
\setbeamercolor{note page}{bg=white}
\setbeamercolor{note title}{bg=white!90!black, fg=black}
\setbeamercolor{note date}{parent=note title}

\beamertemplatenavigationsymbolsempty
\AtBeginSection[]{
    \begin{frame}
        \vfill
        \centering
        \begin{beamercolorbox}[sep=8pt,center,shadow=true,rounded=true]{title}
            \usebeamerfont{title}\insertsectionhead\par%
        \end{beamercolorbox}
        \vfill
    \end{frame}
}

\RequirePackage{listings}
\lstset{escapeinside={<@}{@>}}
\lstset{basicstyle=\ttfamily}
\lstset{moredelim=**[is][\color{red}]{@}{@}}

\RequirePackage{algpseudocode}

\mode<presentation>{}

\title{Rediscovering fire}
\subtitle{On building portable, multi-language libraries}
\author{Jørgen Kvalsvik <jokva@equinor.com>}
\titlegraphic{\includegraphics[width=0.33\textwidth]{equinor-red.eps}}

\begin{document}
\maketitle

\begin{frame}{Outline}
    \tableofcontents

    \note{
        \begin{itemize}
            \item The title alludes to this really being a curated list of what
                  the \emph{field} has known for a long time
            \item Short intro, me + company
            \item Developed a few libraries to work with O\&G file formats
            \item segyio, dlisio, ecl - all largely with the same fundamental
                  architecture
            \item New appreciation to \emph{why} other software is designed the
                  way it is: "so \emph{that's} they do it"
        \end{itemize}
    }
\end{frame}

\begin{frame}
    \begin{block}{}
        The examples and lessons are from segyio
    \end{block}

    \begin{block}{}
        \begin{itemize}
            \item segyio is free software, LGPLv3
            \item Available in Debian, Ubuntu, pypi (pip)
            \item Core is C99
            \item Fronts in C++11, Python2, Python3
            \item The de-facto library for reading/writing seismic
        \end{itemize}
    \end{block}

    \begin{block}{}
        github.com/equinor/segyio
    \end{block}
\end{frame}

\begin{frame}
    \begin{block}{}
        \begin{itemize}
            \item Downloads last day: 174
            \item Downloads last week: 1,883
            \item Downloads last month: 11,879
        \end{itemize}
    \end{block}

    https://pypistats.org/packages/segyio

    21.06.2019
\end{frame}

\section{Why}
\note{Let's start with some \emph{whys}}

\begin{frame}
    \begin{block}{The goal}
        \begin{itemize}
            \item A single, fast, portable core
            \item Multiple, \emph{integrated} language fronts
            \item Clean upgrade paths
        \end{itemize}
    \end{block}

    A library for building \emph{libraries}

    \note{
        \begin{itemize}
            \item Differs from direct language bindings - language bindings
                  are \emph{calling} some third-party solution, then parsing
                  the results
            \item segyio runs on different CPUs, including S390 and multiple
                  operating systems
            \item Emphasize \emph{feeling integrated}
            \item \emph{should feel right} on next page
        \end{itemize}
    }
\end{frame}

\begin{frame}
    User-facing library should \emph{feel right}

    \note{
        \begin{block}{}
            Want more than just to essentially invoke some program from Python
            - want to be able to manipulate input and state as you go and
            \emph{have that be effectful}. Simply copying structures for
            post-processing is not enough. A good example of this is numpy, you
            never really notice (other than speed) that it's native code
        \end{block}

        \begin{block}{}
            Also want to be able to maintain the code without changing clients,
            and be forward- and backwards compatible to the largest extent
            possible.

            Then there's style. Make sure to rename and style according to the
            \emph{host} language's conventions. There are already too many
            camel-case libs in Python
        \end{block}

        \begin{block}{}
            Let's dive right into some details and definitions.
        \end{block}
    }
\end{frame}

\section{API + ABI}
\note{
    \begin{block}{}
        In a lot of languages, like all the scripting ones, there is no concept
        of the ABI, because everything is exchanged as source code.  When
        dealing with native code like Fortran, C++, Rust (less so because it's
        always statically linked) binary compatibility between libraries is
        quite important.
    \end{block}

    \begin{block}{}
        Maintaining ABI stability in particular is not always straight-forward,
        it limits the changes you can do. Some things get somewhat encumbered
        by it, but for a good reason - e.g. Qt. Let's have a quick look at what
        it all really means.
    \end{block}
}

\begin{frame}
    \begin{block}{API}
        \begin{itemize}
            \item Application Programming Interface
            \item Function signatures and types
            \item Source compatibility
            \item Compile-time property
        \end{itemize}
    \end{block}

    \note{
        \begin{itemize}
            \item Programs that worked with the old version, work with
                  recompilation with the new version
            \item Some changes maintain API but not ABI compatibility, e.g. int
                  $\rightarrow$ long
        \end{itemize}
    }
\end{frame}

\begin{frame}
    \begin{block}{ABI}
        \begin{itemize}
            \item Application Binary Interface
            \item Data type, size, alignment
            \item Run-time property
        \end{itemize}
    \end{block}

    \note{
        \begin{itemize}
            \item Programs compiled against the old version also work, without
                  recompilation, with the new version
            \item Compatibility is usually only considered inside same
                  toolchain
        \end{itemize}

        Maintaining all of this makes clients happy because their code doesn't
        break, so hopefully they update more, and it makes updating easier,
        just drop in an so/dll
    }
\end{frame}

\begin{frame}
    \begin{block}{Rules for ABI+ABI stability}
        \begin{itemize}
            \item Never remove functions
            \item Never change arguments
            \item Never change structs
            \item Adding is fine
        \end{itemize}
    \end{block}

    \note{
        These are good guidelines, but not a complete rule set. With the three
        nevers, the API+ABI should remain stable
    }
\end{frame}

\section{Design for primitive types}
\note{
    \begin{block}{}
        Let's talk a bit about how we accomplish our goals. It starts with an
        emphasis on designing around primitive types.
    \end{block}

    \begin{block}{}
        First of all, what are they?
    \end{block}
}

\begin{frame}
    \begin{block}{Primitive types}
        \begin{itemize}
            \item int, float, pointer etc
            \item Avoid structs
            \item Easy to work with, size known an universal
        \end{itemize}
    \end{block}

    \note{
        \begin{itemize}
            \item Structs are also sometimes referred to as \emph{aggregate
                  data types}
            \item Structs in the API are nice for C, but puts many requirements
                  on callers. In practice, ints and char* and floats usually
                  cut it
            \item All kinds of FFIs work well with primitive types, but have
                  more problems with structs. Runtimes can usually convert
                  to/from primitives automatically, but structs need to be
                  replicated and force a certain, maybe impractical,
                  representation in the host language
            \item Let's look at an example to understand \textbf{why}
        \end{itemize}
    }
\end{frame}

\begin{frame}[fragile]
    \begin{block}{C}
        \begin{lstlisting}
struct addrinfo {
    int flags;
    int family;
    int protocol;
};
        \end{lstlisting}
    \end{block}

    \begin{block}{python/ctypes}
        \begin{lstlisting}
import ctypes
class Addrinfo(ctypes.Structure):
    _fields_ = [
        ("flags",    ctypes.c_int),
        ("family",   ctypes.c_int),
        ("protocol", ctypes.c_int),
    ]
        \end{lstlisting}
    \end{block}

    \note{
        \begin{itemize}
            \item Simplified addrinfo from C
            \item Output of getaddrinfo function
            \item Python struct is pretty awkward
            \item But could be hidden in lib so that's ok
        \end{itemize}

        Let's see it in action
    }
\end{frame}

\begin{frame}[fragile]
    \begin{lstlisting}
addrinfo = Addrinfo()
libc.getaddrinfo(ctypes.byref(addrinfo))
    \end{lstlisting}

    \note{
        This Python code creates an empty Addrinfo struct, and passes a pointer
        into it to our getaddrinfo function. This is not so bad, but:
        \begin{itemize}
            \item Forces us to make compatible structs, often creating a
                  mirrored, duplicated interface in Python (with no or little
                  consistency checking!)
            \item Only used for the function call, we probably don't want to
            \item It doesn't feel very integrated, doesn't look like python
            \item If it did, it would've returned an object, taking no args
            \item Must stick it in a lib anyway, so not so bad, but we cannot
                  expose the native functionality directly
            \item The members are ctypes, but want \emph{native}
                  Python, can't keep using cint
        \end{itemize}
    }
\end{frame}

\begin{frame}[fragile]{}
    \begin{block}{}<+->
        \begin{lstlisting}
>>> import ctypes
>>> ctypes.c_int(10).value
10
>>> ctypes.c_int(10) + 5
Traceback (most recent call last):
File "<stdin>", line 1, in <module>
TypeError: unsupported operand type(s) for +:
    'c_int' and 'int'
        \end{lstlisting}
    \end{block}

    \note{
        A lot of other FFIs have the same detail with the .value, although not
        always so verbose. The c-int type is often needed to be able to send
        pointer-to, in order to read the FFIs output. Most FFIs handle
        conversion from host-int to C-int by-value nicely.
    }
\end{frame}

\begin{frame}[fragile]
    \begin{lstlisting}
flags = c_int()
family = c_int()
protocol = c_int()
libc.getaddrinfop(
    ctypes.byref(flags),
    ctypes.byref(family),
    ctypes.byref(protocol)
)
    \end{lstlisting}

    \note{
        \begin{itemize}
            \item No need to make compatible struct
            \item But if caller want struct, easy to make
            \item Multiple vars are essentially structs (aggregate types)
                  anyway, just unpacked. No worse in terms of signatures and
                  ABI
            \item Still awkward because FFIs tend to need special types to get
                  data back. Int-by-value are transformed implicitly, but this
                  does not apply to structs
            \item Python always owns the variable, no lifetime issues
        \end{itemize}
    }
\end{frame}

\begin{frame}[fragile]
    \begin{block}{C}
        \begin{lstlisting}
blob* getdata(int offset, int size);
        \end{lstlisting}
    \end{block}

    \begin{block}{Python}
        \begin{lstlisting}
def load(graph_x, graph_y):
    blob = libc.getdata(offset, size)
    graph_x.add(blob)
    graph_y.add(blob)
        \end{lstlisting}
    \end{block}

Who are responsible for releasing?

    \note{
        \begin{itemize}
            \item Cannot distinguish failures when pointer is returned
            \item Must use something like errno
            \item Can register blob as a type and build a hook in gc.run(), but
                  is undeterministic
        \end{itemize}
    }
\end{frame}

\begin{frame}
    \begin{block}{Integers and gentlemen's agreements}
        \begin{itemize}
            \item Return values
            \item Wide, naturally extensible
            \item $2^{32}$ different values (usually)
            \item Bools are evil, use int for conditionals
        \end{itemize}
    \end{block}

    Enums are integral, but not necessarily int. Use int in signatures, but
    named constants as enums are fine.

    \note{
        So why am I talking so much about ints?
        \begin{itemize}
            \item Integers are great, use them for everything
            \item The int is the same fundamental type, and are quite powerful
                  when paired with \emph{semantics} derived from context. If
                  ints are received and passed along, host code can support new
                  meanings without changing any code.
                  % TODO: example? format
            \item Int for return values are recommendation, will be discussed
                  later
            \item Bools are really ints, but cannot be extended. A lot of
                  parameters start out binary, but later ends up getting more
                  options
            \item When later extending functionality, flags can be combined
            \item Enums are defined to be integrals, not int, so using int in
                  signatures ensures languages without enums ease-of-use and
                  type width
        \end{itemize}
    }
\end{frame}

\section{No allocations}

\begin{frame}
    \begin{block}{Allocation introduce complexity}
        \begin{itemize}
            \item Lifetime and ownership issues
            \item Memory leak or double-free opportunities
            \item new/malloc incompatibilities
        \end{itemize}
    \end{block}

    \note{
        \begin{itemize}
            \item Who owns an object?
            \item Runtime and core may have different ownership models too, and
                  this is difficult to follow
            \item As we saw, when do you free? Need to hook custom destructors
                  into garbage collectors, not deterministic
            \item Forces caller to store C-compatible pointer, which is usually
                  not ergonomic at all
            \item When is it released? Can you copy? Re-init?
            \item Harder to build alternative representations
            \item If you're not allocating in Python, why write Python
        \end{itemize}

        [...] \emph{to not leak is to not allocate} [...] next frame
    }
\end{frame}

\begin{frame}
    \emph{The easiest way to not leak memory is to not allocate}

    \note{
        Is it all lost? What's our alternative
    }
\end{frame}

\begin{frame}[fragile]
    \begin{block}{Alternative: provide two functions}
        \begin{algorithmic}
            \State $n \gets$ \Call{ccall/query-size}{}
            \State $buffer \gets$ \Call{alloc-bytes}{$n$}
            \State \Call{ccall/get-data}{$buffer$}
        \end{algorithmic}
    \end{block}

    \begin{block}{Benefits}
        \begin{itemize}
            \item Style enables paging
            \item Can reuse buffers
            \item Multiple writes in single buffer
            \item Different alloction strategies
            \item Memory registered (and garbage collected) and owned by
                  runtime
        \end{itemize}
    \end{block}

    \note{
        \begin{block}{}
            Can now support different alloc strategies - runtime can pool
            allocations, use arenas, allocate inside other objects, handle
            concurrency. This is effectively forbidden if the core enforces
            allocation a certain way.
        \end{block}

        So what does this look like in practice?
    }
\end{frame}

\begin{frame}[fragile]
    \begin{block}{Python}
        \begin{lstlisting}
buffer = numpy.zeros(shape, dtype)
return self.gettr(buffer, i, 1, 1)
        \end{lstlisting}
    \end{block}

    \begin{block}{Extension (C++)}
        \begin{lstlisting}
PyBuffer buffer;
PyObject_GetBuffer(o, &buffer);
segy_read_trace(i, buffer.data);
        \end{lstlisting}
    \end{block}

    \note{
        Examples of opportunities
        \begin{itemize}
            \item Alloc $shape \cdot n$, call gettr multiple times (makes matrix)
            \item Can use different object, e.g. xarray, BLAS object
            \item When making a generator, re-use same buffer
            \item In general, the library author wants to control memory
        \end{itemize}

        Still suffers from the unpythonic calling style, but this is inside the
        library so it's fine. Could've allocated inside the gettr too, but
        wanted to control that from Python source
    }
\end{frame}

\section{Calling core functions}

\begin{frame}[fragile]
    \begin{lstlisting}
int err = fun(...);
if (err) return err;
    \end{lstlisting}

    \begin{lstlisting}
int err = fun(...);
switch (err) {
    case OK: return OK;
    case e1: handle_e1();
    case e2: handle_e2();
    ...
    default:
        /* always default case */
        return err;
    }
    \end{lstlisting}

    \note{
        \begin{block}{}
            If the runtime invokes the core library in either of these
            fashions, new error codes can be added in the core lib, if more
            edge cases or better descriptions come up later, without changing
            runtime code.
        \end{block}

        \begin{block}{}
            Of course, to make the runtime take advantage of this it must be
            updated, but code won't suddenly misbehave too bad. Still, this is
            a form of \emph{future proofing}.
        \end{block}

        \begin{block}{}
            Arguably, return code is a part of the ABI, and that perspective is
            fine.
        \end{block}
    }

\end{frame}

\begin{frame}[fragile]
    Use int for return type, but choose them from an enum.
    \begin{lstlisting}
typedef enum {
    SEGY_OK = 0,
    SEGY_FREAD_ERROR,
    SEGY_FWRITE_ERROR,
    SEGY_INVALID_FIELD,
    SEGY_INVALID_SORTING,
    SEGY_MISSING_LINE_INDEX,
    SEGY_INVALID_OFFSETS,
    SEGY_TRACE_SIZE_MISMATCH,
    SEGY_INVALID_ARGS,
    SEGY_READONLY,
    SEGY_NOTFOUND,
} SEGY_ERROR;
    \end{lstlisting}
\end{frame}

\section{Primitive functions}
\note{
    Now that we know how to call them, let's look at how to combine the
    no-allocation primitive-type lessons
}

\begin{frame}
    Identify a useful set of \textbf{atoms} that solve the \textbf{underlying
    problem}

    \note{
        \begin{block}{}
            This is really just good software design - functions should have a
            single responsibility anyway.
        \end{block}

        \begin{block}{}
            The one thing I've noticed that I'd like, even with our super low
            level interface, is to have \emph{even} more details exposed. When
            they don't work they can be augmented by more or slightly different
            functions, or runtime-specific changes.
        \end{block}

        \begin{block}{}
            Remember, \emph{your users want to write libraries}, hiding details
            is likely harmful. If two functions can be combined to provide the
            functionality, there's no reason to introduce a third
        \end{block}
    }
\end{frame}

\begin{frame}
    \begin{block}{Function design guidelines}
        \begin{itemize}
            \item All state through parameters, outside-in wiring
            \item Heavy parametrized
            \item Return int error code
            \item Once assembled, the core should solve the problem
            \item Resist the temptation of \emph{solving} it in Python
        \end{itemize}
    \end{block}

    \note{
        \begin{itemize}
            \item It makes it very clear what data is needed, allowing the
                  consumer to initialise and prepare (cache, multi-phase
                  initialise) accordingly
            \item By parametrizing, feature selection can be done by storing
                  variables, less need for dispatch tables
            \item Also allows overriding at every step. Case study: take index
                  as parameter, allows storing and reading from disk. Doing
                  this implicitly good in short term, harmful in long term
            \item Example: can't manipulate trace index from Python unless
                  index is fully exposed, can't store as Python sees fit
            \item Wiring is done from the outside, they have more
                  \textbf{context}. Easy to report error on \emph{exactly} the
                  step that failed
            \item Because they're small in function and scope they're easy to
                  test. Errors can be simulated by passing wrong parameters,
                  and, setup can be omitted
        \end{itemize}
    }
\end{frame}

\begin{frame}[fragile]
    \begin{block}{Example: State through parameters}
        \begin{lstlisting}
int segy_readsubtr(segy_file* fp,
                   int traceno,
                   int start,
                   int stop,
                   int step,
                   void* buf,
                   void* rangebuf,
                   long trace0,
                   int trace_bsize);
        \end{lstlisting}
    \end{block}

    \note{
        \begin{block}{}
            This is one of the workhorse functions in segyio. If you squint,
            the functionality is really reading a (possibly strided) column of
            a very large matrix
        \end{block}
    }
\end{frame}

\begin{frame}[fragile]
    \begin{block}{Example: State through parameters}
    \begin{lstlisting}
int segy_readsubtr(segy_file* fp,
                   int traceno,
                   @int start,@
                   @int stop,@
                   @int step,@
                   void* buf,
                   void* rangebuf,
                   long trace0,
                   int trace_bsize);
        \end{lstlisting}
    \end{block}

    \note{
        \begin{block}{}
            These parameters provide the strided read. Reading a full column is
            just a special case of this.
        \end{block}

        \begin{block}{}
            Now callers can dispatch both strided reads and full columns
            through parameters, rather than having to dispatch to different
            functions
        \end{block}
    }
\end{frame}

\begin{frame}[fragile]
    \begin{block}{Example: State through parameters}
        \begin{lstlisting}
int segy_readsubtr(segy_file* fp,
                   int traceno,
                   int start,
                   int stop,
                   int step,
                   void* buf,
                   void* rangebuf,
                   @long trace0,@
                   @int trace_bsize@);
        \end{lstlisting}
    \end{block}

    \note{
        These parameters would often be inside the fp handle. They mark the
        column size (in bytes), and the start of the data. This has proven very
        useful to keep passing manually, even though they're \emph{usually} the
        same values all the time. They're provided by the \emph{library
        itself}, Python never even inspects them

        \begin{itemize}
            \item It powers reading-from-offsets, essentially parallel
                  matrices (or the next dimension)
            \item If set in fp, to move start-of-file, you must first push the
                  new start, do the read, and pop it for one-offs
            \item Callers can re-interpret traces, allowed easy extension to
                  2/3/8-byte formats. Virtually no code changed in Python
            \item Different behaviours could easily be simulated
        \end{itemize}

        Let's look at an alternative
    }
\end{frame}

\begin{frame}[fragile]
    \begin{lstlisting}
int segy_to_native(int format,
                   long long size,
                   void* buf);
    \end{lstlisting}
    \note{
        \begin{itemize}
            \item ibm float, ieee 4+8 byte
            \item signed integer 1-2-3-4-8 byte
            \item unsigned integer 1-2-3-4-8 byte
            \item \texttt{format} encodes little/big endian
        \end{itemize}

        Now, if this was without the format parameter
    }
\end{frame}

\begin{frame}[fragile]
    \begin{lstlisting}
int segy_from_ieee4be(int size, float* buf);
int segy_from_ieee8be(int size, double* buf);
int segy_from_ibm4be(int size, float* buf);
int segy_from_int4be(int size, float* buf);

int segy_from_ieee4le(int size, float* buf);
int segy_from_ieee8le(int size, double* buf);
int segy_from_ibm4le(int size, float* buf);
int segy_from_int4le(int size, float* buf);
    \end{lstlisting}

    \note{
        And similarly for native-to-file. Now the same information is encoded
        in the function name, but forces host language to implement the mapping
        logic, instead of storing an int it gets from somewhere else.
    }
\end{frame}

\begin{frame}
    Separate I/O from all other logic

    \note{
        \begin{itemize}
            \item This is one of those ancient programming advices that still is
                  easy to mess up.
            \item segyio is an I/O library. What's the gain in not providing
                  I/O?  How well can I/O be separated?
            \item Virtually all segyio functions operate on an opaque file
                  handle, which I think was a mistake
            \item What is "I/O" really?
        \end{itemize}
    }
\end{frame}

\begin{frame}[fragile]
    \begin{lstlisting}
int segy_to_native(int format,
                   long long size,
                   void* buf);
    \end{lstlisting}

    \begin{itemize}
        \item This is I/O (memory transaction)
        \item Disconnected from file reading
        \item Interface supports arbitrary element sizes
        \item Can translate partial buffers
        \item Source of data does not have to be SEG-Y
    \end{itemize}

    \note{
        \begin{block}{}
            Maybe single best design decision of segyio, can encode/decode in
            batch, read raw data etc. Anecdotally, new numeric format support
            was added without any change in Python, silently accepted more
            files.
        \end{block}

        \begin{block}{}
            Enables good \emph{integration}. Very much applies to file I/O,
            e.g. read directly from cloud, mmap, unix pipes etc. segyio today
            uses an opaque file handle, which prohibits files in the cloud
        \end{block}
    }
\end{frame}

\begin{frame}
    \vfill
    \centering
    \begin{beamercolorbox}[sep=8pt,center,shadow=true,rounded=true]{title}
        \usebeamerfont{title}Thank you!
    \end{beamercolorbox}
    \vfill
\end{frame}

\begin{frame}
    \vfill
    \centering
    \begin{beamercolorbox}[sep=8pt,center,shadow=true,rounded=true]{title}
        \usebeamerfont{title}Questions?
    \end{beamercolorbox}
    \vfill
\end{frame}

\end{document}
