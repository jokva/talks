\documentclass[pdf]{beamer}

\RequirePackage[utf8]{inputenc}
\RequirePackage[T1]{fontenc}
\RequirePackage{lmodern}

% for speaker notes etc
\RequirePackage{pgfpages}
% remove this line to only get slides
\setbeameroption{show notes on second screen}

\setbeamercovered{invisible}
\setbeamercovered{again covered={\opaqueness<1->{15}}}

\AtBeginSection[]{
    \begin{frame}
        \vfill
        \centering
        \begin{beamercolorbox}[sep=8pt,center,shadow=true,rounded=true]{title}
            \usebeamerfont{title}\insertsectionhead\par%
        \end{beamercolorbox}
        \vfill
    \end{frame}
}

\RequirePackage{listings}

\mode<presentation>{}

\title{Rediscovering fire}
\author{Jørgen Kvalsvik}


\begin{document}
\maketitle

\begin{frame}{Outline}
    \tableofcontents
\end{frame}

\section{Why}

\begin{frame}{Goal}
    % Drawing of single-core multi-front
    \begin{block}{The goal}
        A single, fast core that can be shared by multiple language fronts,
        with clean upgrade paths. A library for building \emph{libraries}.
    \end{block}
    \note{
        Differs from direct language bindings
    }
\end{frame}

\section{API + ABI}

\begin{frame}
    \begin{block}{API}
        \pause
        \begin{itemize}[<+>]
            \item Application Programming Interface
            \item Function signatures and types
            \item Source compatibility
            \item Compile-time property
        \end{itemize}
    \end{block}

    \note{
    \begin{itemize}
        \item Programs that worked with the old version, work with
              recompilation with the new version
        \item Some changes maintain API compatibility, e.g. int $\rightarrow$ long
    \end{itemize}
    }
\end{frame}

\begin{frame}

        \begin{block}{ABI}
        \pause
        \begin{itemize}[<+>]
            \item Application Binary Interface
            \item Data type, size, alignment
            \item Run-time property
        \end{itemize}
    \end{block}

    \note{
    \begin{itemize}
        \item Programs compiled against the old version also work, without
              recompilation, with the new version
        \item Compatibility is usually only considered inside same toolchain
    \end{itemize}
    }
\end{frame}

\begin{frame}
    \begin{block}{Rules for ABI+ABI stability}
    \pause
        \begin{itemize}[<+>]
            \item Never remove functions
            \item Never change arguments
            \item Never change structs
            \item Adding is fine
        \end{itemize}
    \end{block}

    \note{
        These are good guidelines, but not a complete rule set. With the three
        nevers, the API+ABI should remain stable
    }
\end{frame}

\section{Primitive types}
\begin{frame}
    \begin{block}{Primitive types}
        \begin{itemize}
            \item int, float, pointer etc
            \item Avoid structs
        \end{itemize}
    \end{block}

    \note{
        Structs in the API are nice for C, but puts many requirements on
        callers.

        Let's look at \textbf{why}.
    }
\end{frame}

\begin{frame}[fragile]
    \begin{block}{C}<+>
        \begin{lstlisting}
struct addrinfo {
    int flags;
    int family;
    int protocol;
};
        \end{lstlisting}
    \end{block}

    \begin{block}{python/ctypes}<+>
        \begin{lstlisting}
import ctypes
class Addrinfo(ctypes.Structure):
    _fields_ = [
        ("flags",    ctypes.c_int),
        ("family",   ctypes.c_int),
        ("protocol", ctypes.c_int),
    ]
        \end{lstlisting}
    \end{block}

    \note{
        \begin{itemize}
            \item Simplified addrinfo from C
            \item Output of getaddrinfo function
            \item Python struct is pretty awkward
            \item But could be hidden in lib so that's ok
        \end{itemize}

        Let's see it in action
    }
\end{frame}

\begin{frame}[fragile]
    \begin{lstlisting}
addrinfo = Addrinfo()
libc.getaddrinfo(ctypes.byref(addrinfo))
    \end{lstlisting}

    \note{
        This is not so bad, but:
        \begin{itemize}
            \item Forces us to make compatible structs
            \item The members are ctypes
            \item We want \emph{native} Python, can't keep using cint
        \end{itemize}
    }
\end{frame}

\begin{frame}[fragile]{}
    \begin{block}{}<+>
        \begin{lstlisting}
>>> import ctypes
>>> ctypes.c_int(10) + 5
Traceback (most recent call last):
File "<stdin>", line 1, in <module>
TypeError: unsupported operand type(s) for +:
    'c_int' and 'int'
        \end{lstlisting}
    \end{block}

    \begin{block}{}<+>
        \begin{lstlisting}
>>> ctypes.c_int(10).value
10
        \end{lstlisting}
    \end{block}

    \note{
    }
\end{frame}

\begin{frame}[fragile]
    \begin{lstlisting}
flags = c_int()
family = c_int()
protocol = c_int()
libc.getaddrinfop(
    ctypes.byref(flags),
    ctypes.byref(family),
    ctypes.byref(protocol)
)
    \end{lstlisting}

    \note{
        \begin{itemize}
            \item No need to make compatible struct
            \item But if caller want struct, easy to make
            \item Multiple vars are essentially structs (aggregate types)
                  anyway, if ABI to be maintained
            \item In some languages, cint isn't special, and can write directly
            \item Python always owns the variable, no lifetime issues
        \end{itemize}
    }
\end{frame}

\begin{frame}
    \begin{block}{Integers and gentlemen's agreements}
        \pause
        \begin{itemize}[<+>]
            \item Return values
            \item Wide, naturally extensible
            \item $2^{32}$ different values (usually)
            \item Use for optionals, bools are evil
        \end{itemize}
    \end{block}

    \pause
    Named int constants can go in headers as enum,
    Provide int constants in headers as enum, use int in signatures

    \note{
        \begin{itemize}
            \item Integers are great, use them for this
            \item Return values are recommendation, will be discussed later
            \item Bools are really ints, but cannot be extended
            \item When later extending functionality, flags can be combined
        \end{itemize}
    }
\end{frame}

\section{Allocation}

\begin{frame}
    \begin{block}{Allocation introduce complexity}
        \pause
        \begin{itemize}[<+>]
            \item Lifetime issues
            \item malloc/free mismatch
            \item new/malloc mismatch
        \end{itemize}
    \end{block}

    \note{
        \begin{itemize}
            \item Who owns an object?
            \item When is it released? Can you copy? Re-init?
            \item Units can't just receive objects, must know origin
            \item If you're allocating in C, why not write C
        \end{itemize}

        [...] \emph{to not leak is to not allocate} [...] next frame
    }
\end{frame}

\begin{frame}
    \emph{The easiest way to not leak memory is to not allocate}
\end{frame}

\begin{frame}
    \begin{block}{Alternative: two functions}<+->
        \begin{enumerate}[<+->]
            \item Query size
            \item Let caller allocate
            \item Write into callers buffer
        \end{enumerate}
    \end{block}

    \begin{block}{Benefits}<+->
        \begin{itemize}[<+>]
            \item Style enables paging
            \item Can reuse buffers
            \item Multiple writes in single buffer
            \item Different alloc strategies
            \item Runtime owns memory, not core
        \end{itemize}
    \end{block}

    \note{
    }
\end{frame}

\begin{frame}[fragile]
    \begin{block}{Python}<+>
        \begin{lstlisting}
buffer = numpy.zeros(shape, dtype)
return self.gettr(buffer, i, 1, 1)
        \end{lstlisting}
    \end{block}

    \begin{block}{Extension (C++)}<+>
        \begin{lstlisting}
PyBuffer buffer;
PyObject_GetBuffer(o, &buffer);
segy_read_trace(i, buffer.data);
        \end{lstlisting}
    \end{block}

    \note{
        Examples of opportunities
        \begin{itemize}
            \item Alloc shape * n, call gettr multiple times (makes matrix)
            \item Can use different object, e.g. xarray, BLAS object
            \item When making a generator, re-use same buffer
            \item In general, the library author wants to control memory
        \end{itemize}
    }
\end{frame}

\section{Style}

\begin{frame}[fragile]
    \begin{lstlisting}
        int err = fun(...);
        if (err) return err;
    \end{lstlisting}

    \begin{lstlisting}
        int err = fun(...);
        switch (err) {
            case OK: return OK;
            case e1: handle_e1();
            case e2: handle_e2();
            ...
            default:
                /* always default case */
                return err;
        }
    \end{lstlisting}

\end{frame}

\section{Single purpose functions}

\begin{frame}[fragile]
    \begin{block}{}<+>
        \begin{lstlisting}
            int solve(args...)
        \end{lstlisting}
    \end{block}

    EXPOSE YOUR STEPS

    \note{
        It's tempting to make functions like int solve(args...)

        \begin{itemize}
            \item As end user, it is all you care about
            \item As library writer you want something more
        \end{itemize}

        This is really just good software design - functions should have a
        single responsibility anyway. Yet extra care must be applied because
        things added to API are permanent.

        Callers can write solve

        Step2 can be replaced when there's a bug, step1..3 can be re-purposed
        TODO: highly parametrised
        TODO: individually report error, get context from function itself
        TODO: less strict on input, step2 can be substitued
        TODO: don't know the host languages assumptions and opinions
    }
\end{frame}

\begin{frame}[fragile]
    \begin{block}{}<+>
        \begin{lstlisting}
            step1(...)
            step2(...)
            step3(...)
        \end{lstlisting}
    \end{block}

    \note{}
\end{frame}

\begin{frame}
\end{frame}

\end{document}
