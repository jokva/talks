\documentclass[pdf]{beamer}

\setbeamercovered{invisible}
\setbeamercovered{again covered={\opaqueness<1->{15}}}

\mode<presentation>{}

\RequirePackage[utf8]{inputenc}
\RequirePackage[T1]{fontenc}
\RequirePackage{ae,aecompl}

%\RequirePackage{color}
%\definecolor{codegreen}{rgb}{0,0.6,0}
%\definecolor{codegray}{rgb}{0.5,0.5,0.5}
%\definecolor{codepurple}{rgb}{0.58,0,0.82}
%\definecolor{backcolour}{rgb}{0.95,0.95,0.92}
%
%\RequirePackage{listings}
%\lstdefinestyle{python}{
%    language = Python,
%    commentstyle=\color{codegreen},
%    keywordstyle=\color{magenta},
%    numberstyle=\tiny\color{codegray},
%    stringstyle=\color{codepurple},
%    basicstyle=\footnotesize,
%    breakatwhitespace=false,
%    breaklines=true,
%    captionpos=b,
%    keepspaces=true,
%    showspaces=false,
%    showstringspaces=false,
%    showtabs=false,
%    tabsize=2
%}
%\lstset{style = Python}
%
%\RequirePackage{siunitx}
%\DeclareSIUnit\year{yr}
%
%\RequirePackage{adjustbox}
%\RequirePackage{graphics}
%
%\RequirePackage{algorithmicx}

\title{Working with git}
\author{Jørgen Kvalsvik}
\date{January 2019}

\begin{document}
\maketitle

\begin{frame}
    \frametitle{Outline}
    \tableofcontents
\end{frame}

\section{Basic concepts}
\tableofcontents[currentsection]

\begin{frame}{The commit}
    At it's core, git is a key-value store

    \begin{description}
        \item [Key] Object hash
        \item [Value] Tree snapshot + parent pointers
    \end{description}

    \pause
    The commit is \emph{immutable}, any change is a new \emph{snapshot}, and a
    new \emph{hash}
\end{frame}

\begin{frame}[fragile]{The history}
    Because of the parent pointer, the \textbf{commit} knows all previous
    snapshots (by chasing parents) all the way to the \emph{initial commit}

    \begin{block}
        {A <-- B <-- C <-- D master}
    \end{block}
\end{frame}

\begin{frame}[fragile]{The branch}
    \begin{block}{}<+>
        Because commits are immutable, \textbf{branches} are cheap.
        Essentially, they're \emph{movable} pointers
    \end{block}

    \begin{block}{}<+>
        \verb|git commit| moves the pointer of your \emph{current} branch (called
        HEAD) one commit
    \end{block}
\end{frame}

\begin{frame}[fragile]{The merge}
The \textbf{merge} is combining a set of \emph{commits (hash + snapshot +
pointers)} into a new \emph{commit}

\begin{verbatim}
      A---B---C topic 1
     /         \
D---E---F---G---J master
         \     /
          H---I topic 2
\end{verbatim}
\end{frame}

\begin{frame}[fragile]{The rebase}
The \textbf{rebase} is \emph{reapplying} a set of commits onto a new tip

    \begin{block}{}<+>
        \begin{verbatim}
      A---B---C topic
     /
D---E---F---G---J master \end{verbatim}
    \end{block}
    \begin{block}{}<+>
    \begin{verbatim}
                  A---B---C master + topic
                 /
D---E---F---G---J \end{verbatim}

        Logically, A was developed after J
    \end{block}
\end{frame}

\begin{frame}[fragile]{The rebase}
    \begin{quote}
        What and why?

        \verb|git rebase HEAD^5|
    \end{quote}
\end{frame}

\end{document}
