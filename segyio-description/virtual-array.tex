\begin{tikzpicture}[>=latex]
\tikzset{
mymat/.style={
  matrix of math nodes,
  text height=2.5ex,
  text depth=0.75ex,
  text width=3.25ex,
  align=center,
  column sep=-\pgflinewidth,
  style={nodes=draw}
  },
mymats/.style={
  matrix,
  align=center,
  style={nodes={draw, minimum height=4cm}},
  column sep=4\pgflinewidth
  }
}

\matrix [mymat, anchor=west] at (0,0)
(ftrace)
{
    0 & 1 & 2 & \node[draw=none, text width=4ex] {\dots}; & n \\
};
\matrix [mymats, anchor=west] at (0,-4)
(segy)
{
    \node [fill=gray!20, text width=8ex] {\rotatebox{90}{textual file header}}; &
    \node [fill=gray!20, text width=3ex] {\rotatebox{90}{binary file header}};  &
    \node (h1) [text width=3ex]          {\rotatebox{90}{1st trace header}}; &
    \node (t1) [text width=8ex]          {\rotatebox{90}{1st trace}}; &
    \node (h2) [text width=3ex]          {\rotatebox{90}{2nd trace header}}; &
    \node (t2) [text width=8ex]          {\rotatebox{90}{2nd trace}}; &
    \node (h3) [text width=3ex]          {\rotatebox{90}{3rd trace header}}; &
    \node (t3) [text width=8ex]          {\rotatebox{90}{3rd trace}}; &
    \node [draw=none, text width=5ex]    {$\dots$}; &
    \node (hn) [text width=3ex]          {\rotatebox{90}{$n$th trace header}}; &
    \node (tn) [text width=8ex]          {\rotatebox{90}{$n$th trace}}; \\
};

\node[left=0pt of ftrace](ftracelabel) {\verb|f.trace|};
\node[left=0pt of segy](filelabel) {SEG-Y};

\begin{scope}[shorten <= -2pt]
    \draw[*->] (ftrace-1-1.south) to[looseness=0, out=-90] (t1.north west);
    \draw[*->] (ftrace-1-2.south) to[looseness=0, out=-90] (t2.north west);
    \draw[*->] (ftrace-1-3.south) to[looseness=0, out=-90] (t3.north west);
    \draw[*->] (ftrace-1-5.south) to[looseness=0, out=-90] (tn.north west);
\end{scope}

\end{tikzpicture}
